\documentclass[a4paper]{scrartcl}

% font/encoding packages
\usepackage[utf8]{inputenc}
\usepackage[T1]{fontenc}
\usepackage{lmodern}
\usepackage[ngerman]{babel}
\usepackage[ngerman=ngerman-x-latest]{hyphsubst}

\usepackage{amsmath, amssymb, amsfonts, amsthm}
\usepackage{array}
\usepackage{stmaryrd}
\usepackage{marvosym}
\allowdisplaybreaks
\usepackage[output-decimal-marker={,}]{siunitx}
\usepackage[shortlabels]{enumitem}
\usepackage[section]{placeins}
\usepackage{float}
\usepackage{units}
\usepackage{listings}
\usepackage{pgfplots}
\pgfplotsset{compat=1.12}

\newtheorem*{behaupt}{Behauptung}
\newcommand{\gdw}{\Leftrightarrow}
\newcommand{\N}{\mathbb{N}}
\newcommand{\cov}{\operatorname{Cov}}
\newcommand{\e}{\operatorname{E}}
\newcommand{\var}{\operatorname{Var}}
\newcommand{\corr}{\operatorname{Corr}}

\usepackage{fancyhdr}
\pagestyle{fancy}

\def \blattnr {1 }

\lhead{Stochastik 2 - Blatt \blattnr}
\rhead{Florian Abt, Lennart Braun, TODO}
\cfoot{\thepage}


\title{Stochastik 2 für Informatiker}
\subtitle{Blatt \blattnr Hausaufgaben}
\author{
    Florian Abt (6524404), \\
    Lennart Braun (6523742), \\
    TO DO (TODO)
}
\date{zum 20. Oktober 2015}

\begin{document}
\maketitle

\begin{enumerate}[label=\bfseries 1.\arabic*]
    \item
        \begin{enumerate}[label=\alph*)]
            \item

            \item

        \end{enumerate}

    \item
        \begin{enumerate}[label=\alph*)]
            \item

            \item

            \item

        \end{enumerate}

    \item
        \begin{enumerate}[label=\alph*)]
            \item

            \item

        \end{enumerate}

    \item
        Die Zufallsvariable $W$ habe Werte in $\mathbb{N}$ und erfülle
        $\mathbb{P}(W = k \ |\  W > n) = \mathbb{P}(W = k-n)$ für alle
        $n \in \mathbb{N}_0$ und für alle $k \in \{n+1, n+2, \dotsc\}$.

        \begin{behaupt}
            Es gilt $W \sim \text{Geom}_p$ mit $p \in (0,1]$.
        \end{behaupt}
        \begin{proof}
            TODO
        \end{proof}

\end{enumerate}


\end{document}
