\documentclass[a4paper]{scrartcl}

% font/encoding packages
\usepackage[utf8]{inputenc}
\usepackage[T1]{fontenc}
\usepackage{lmodern}
\usepackage[ngerman]{babel}
\usepackage[ngerman=ngerman-x-latest]{hyphsubst}

\usepackage{amsmath, amssymb, amsfonts, amsthm}
\usepackage{array}
\usepackage{stmaryrd}
\usepackage{marvosym}
\usepackage{hhline}
\allowdisplaybreaks
\usepackage[output-decimal-marker={,}]{siunitx}
\usepackage[shortlabels]{enumitem}
\usepackage[section]{placeins}
\usepackage{float}
\usepackage{units}
\usepackage{listings}
\usepackage{pgfplots}
\pgfplotsset{compat=1.12}

\newtheorem*{behaupt}{Behauptung}
\newcommand{\gdw}{\Leftrightarrow}
\newcommand{\N}{\mathbb{N}}
\newcommand{\cov}{\operatorname{Cov}}
\newcommand{\e}{\operatorname{E}}
\newcommand{\var}{\operatorname{Var}}
\newcommand{\corr}{\operatorname{Corr}}

\usepackage{fancyhdr}
\pagestyle{fancy}

\def \blattnr {1 }

\lhead{Stochastik 2 - Blatt \blattnr}
\rhead{Florian Abt, Lennart Braun, TODO}
\cfoot{\thepage}


\title{Stochastik 2 für Informatiker}
\subtitle{Blatt \blattnr Hausaufgaben}
\author{
    Florian Abt (6524404), \\
    Lennart Braun (6523742), \\
    TO DO (TODO)
}
\date{zum 20. Oktober 2015}

\begin{document}
\maketitle

\begin{enumerate}[label=\bfseries 1.\arabic*]
    \item
        \begin{enumerate}[label=\alph*)]
            \item

            \item

        \end{enumerate}

    \item
        \begin{enumerate}[label=\alph*)]
            \item

            \item

            \item

        \end{enumerate}

    \item
        \begin{enumerate}[label=\alph*)]
            \item Die erste Markov-Eigenschaft ist erfüllt, da im n+1-ten  Wurf entweder eine 6 gewürfelt wird oder nicht. Die Wahrscheinlichkeit für $ X_{n+1} = i_{n+1} $ hängt also nur vom Wert von $ X_{n} $  ab. Die Werte von Zufallsvariablen früherer Würfe haben auf die Wahrscheinlichkeit des n+1-ten Wurfs  keine Auswirkung, folglich gilt:\\\\
            $ \mathbb{P} \left( X_{n+1}=i_{n+1}|X_{n}=i_{n},X_{n-1}=i_{n-1},...,X_{0}=i_{0}\left) = \mathbb{P} \left( X_{n+1}=i_{n+1}|X_{n}=i_{n} \left) $\\\\
            Die zweite Markov-Eigenschaft ist nicht erfüllt. Zwar ist die Wahrscheinlichkeit eine 6 zu würfeln in jedem Wurf gleich groß, jedoch sind die Einschritt-Übergangswahrscheinlichkeiten nicht immer gleich. Sie hängen von n ab, denn $ X_{n} $ kann nur Werte annehmen, die kleiner gleich n sind, folglich gilt also nicht:\\\\
            $ \mathbb{P} \left( X_{n+1}=j | X_{n} \left) = \mathbb{P} \left( X_{1} = j | X_{0} = i \left)$ \\\\
            Es liegt keine Homogenität vor und damit ist $ (X)_{n\in\mathbb{N}} $ keine DTMC. 
            \item Die erste Markov-Eigenschaft ist erfüllt, da im n+1-ten  Wurf entweder eine Zahl gewürfelt wird, die bisher die größte ist, oder nicht. Die Wahrscheinlichkeit für $ X_{n+1} = i_{n+1} $ hängt also nur vom Wert von $ X_{n} $ ab. Die Ergebnisse früherer Würfe haben auf die Wahrscheinlichkeit des n+1-ten Wurfs  keine Auswirkung, folglich gilt:\\\\
            $ \mathbb{P} \left( X_{n+1}=i_{n+1}|X_{n}=i_{n},X_{n-1}=i_{n-1},...,X_{0}=i_{0}\left) = \mathbb{P} \left( X_{n+1}=i_{n+1}|X_{n}=i_{n} \left) $\\\\
            Die zweite Markov-Eigenschaft ist erfüllt. Die Wahrscheinlichkeit im n+1-ten Wurf eine Zahl größer i zu würfeln ist genau so groß wie in jedem anderen Wurf. Die Einschritt-Übergangswahrscheinlichkeiten ist also immer gleich und hängt nicht von n ab. Folglich gilt:\\\\
            $ \mathbb{P} \left( X_{n+1}=j | X_{n} \left) = \mathbb{P} \left( X_{1} = j | X_{0} = i \left)$ \\\\
            Es liegt Homogenität vor und damit ist $ (X)_{n\in\mathbb{N}} $ eine DTMC.\\
            Zustandsraum: $ \Omega = \{1,2,3,4,5,6\} $\\
            Die Einschritt-Übergangs-Wahrscheinlichkeit:\\
            \begin{tabular}[c]{|c||c|c|c|c|c|c|}
            \hline
			$ j\backslash i $&1&2&3&4&5&6\\\hhline{|=||=|=|=|=|=|=|}
			1&$ \dfrac{1}{6} $&0&0&0&0&0\\\hline
			2&$ \dfrac{1}{6} $&$ \dfrac{2}{6} $&0&0&0&0\\\hline
			3&$ \dfrac{1}{6} $&$ \dfrac{1}{6} $&$ \dfrac{1}{2} $&0&0&0\\\hline
			4&$ \dfrac{1}{6} $&$ \dfrac{1}{6} $&$ \dfrac{1}{6} $&$ \dfrac{2}{3} $&0&0\\\hline
			5&$ \dfrac{1}{6} $&$ \dfrac{1}{6} $&$ \dfrac{1}{6} $&$ \dfrac{1}{6} $&$ \dfrac{5}{6} $&0\\\hline
			6&$ \dfrac{1}{6} $&$ \dfrac{1}{6} $&$ \dfrac{1}{6} $&$ \dfrac{1}{6} $&$ \dfrac{1}{6} $&1\\\hline
			\end{tabular}
             
            	

        \end{enumerate}

    \item
        Die Zufallsvariable $W$ habe Werte in $\mathbb{N}$ und erfülle
        $\mathbb{P}(W = k \ |\  W > n) = \mathbb{P}(W = k-n)$ für alle
        $n \in \mathbb{N}_0$ und für alle $k \in \{n+1, n+2, \dotsc\}$.

        \begin{behaupt}
            Es gilt $W \sim \text{Geom}_p$ mit $p \in (0,1]$.
        \end{behaupt}
        \begin{proof}
            TODO
        \end{proof}

\end{enumerate}


\end{document}
